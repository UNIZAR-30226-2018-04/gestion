\section{Plan de gestión del proyecto}
\label{planes}
\subsection{Procesos}
\label{procesos}

El proyecto está dividido en dos iteraciones, cada una de ellas posee cinco fases: requisitos, análisis, diseño, implementacion y pruebas. En la primara iteración se va a desarrollar el funcionamiento de una partida tanto individual como por parejas y la página web con las pantallas de login y perfil de usuario. En la segunda iteración se va a desarrollar el resto de funcionalidades del sistema, que incluyen la inteligencia artificial, torneos, tienda, entre otros.  Cada fase de cada iteración tendrá una duración máxima de una semana, excepto la fase de recogida de requisitos que será de duración inferior y que solo se llevará acabo al inicio del proyecto.
A lo largo del proyecto el equipo se dividirá en los grupos de desarrollo: bases de datos y despliegue, backend, frontend y middelware e IA. Donde la participación será flexible pero que mantendrá de forma fija a un responsable.

\subsubsection{Procesos de inicio del proyecto}
\label{inicio}

Al iniciar el proyecto, cada grupo de desarrollo realizará y ejecutará un plan de adquisición de aquellos recursos necesarios para el desarrollo de su objetivo. Dichos recursos se limitarán a aquellos que figuran en la propuesta económica, a excepción de recursos extra que considere cada responsable de grupo. Estos recursos extra, si poseen un coste economico, deberán ser presentados y justificados al director del proyecto, el cual aprobará o rechazará.

En la propuesta económica figura únicamente la contratación de un servicio cloud en Amazon AWS y la adquisición de un domio web para acceder a la aplicación. El equipo de bases de datos y despliegue completará el registro y configuración de una cuenta en Amazon y realizará un estudio para finalmente adquirir un dominio web. Al finalizar el proyecto estos recursos serán traspasados al cliente. El resto de recursos utilizados qué no aparecen en la propuesta técnica y económica, poseen una licencia de uso libre y por lo tanto no suponen un coste a la empresa, por lo que no aparecen justificados en la propuesta económica. Estos recursos incluyen diferentes IDEs para el desarrollo de la aplicación y un certificado TLS gratuito obtenido através del servicio "Let's Encrypt". Cada equipo gestionará la adquisición de sus recursos.

Dado que que el proyecto integra un gran número de tecnologías y capas que conforman la arquitectura web, es necesario realizar un estudio de viabilidad de las tecnologías escogidas antes de realizar la adquisición del software de desarrollo, este estudio requerirá de 5 horas de trabajo entre todos los grupos.  En un principio el desarrollo del proyecto se realizará sobre JavaEE, JSP, WebSockets, JavaScript, Bootstrap y MySQL. Dado que una parte del equipo de desarrollo posee cierta experiencia en el desarrollo web, no tanto en la utilización de estas herramientas, es necesario que se lleve a cabo un proceso de aprendizaje inicial. Este proceso requerirá número mínimo de horas que peprmitirá al equipo comenzar con el desarrollo del proyecto. No obstante a lo largo del proyecto la formación y la adquisición de experiencia serán fundamentales.\\
En aquellas tecnologías donde el equipo de desarrollo no posee ninguna experiencia, como WebSockets, JavaScript , JSP,  entre otros, se requerirá un esfuerzo extra por parte del equipo para auto-formarse mediante la lectura de libros, tutoriales y la documentación pertinente.
\begin{itemize}
	\item Tanto Carlos Marañés como Javier Corbalán se formarán en Javascript, Websockets y Phaser.io a través del uso de ejemplos y documentación en línea.
	\item Julia Guerrero y Sergio Izquierdo se formarán en HTML/CSS y JSP con documentación en línea y  con la ayuda de los miembros del equipo que tienen experiencia en esa tecnología.
    \item El equipo entero se formará en la dinámica de trabajo con AWS gracias a la experiencia de Sergio Izquierdo y tutoriales en línea.
\end{itemize}

\subsubsection{Procesos de ejecución y control del proyecto}

Una de las funciones más importantes del director del proyecto es gestionar la comunicación dentro del equipo. Para garantizar que esta máxima se cumple, el director tendrá la capacidad de convocar reuniones que incluyan a todo el equipo o solo a los responsables de cada grupo de desarrollo. En cada reunión se realizará un acta que recoja todos los aspectos y decisiones importantes acaecidos en la reunión. Las decisiones tomadas en estas reuniones deberán trasladarse a la documentación del proyecto y finalmente a la implementación.
\\\\
Los grupos de desarrollo se coordinarán de forma autónoma para no sobrecargar la figura de director, para ello existirá la figura de responsable, ya avanzada anteriormente. Si el equipo de trabajo sigue un correcto funcionamiento el director solo deberá reunirse con los responsables de cada grupo, sin embargo en ocasiones extraordinarias podrá reunirse con todo el grupo para tomar decisiones que engloben a todo el proyecto, o para corregir posibles malfuncionamientos. Los objetivos globales de cada grupo serán supervisados semana a semana por el director del proyecto. Mientras que dentro de cada grupo el responsable asignará día a día las tareas necesarias para cumplir con los objetivos. Los objetivos que semanalmente cada grupo de desarrollo se marca deben ser sencillos, concretos y trazables, de forma que permitan  medir semanalmente el progreso del proyecto. Si una semana un grupo de desarrollo no cumple con las tareas asignadas el responsable puede reforzar el grupo con nuevos miembros y enfocando las tareas de forma diferente para completar esos objetivos antes de la semana siguiente.

En aquellas situaciones donde se requiera mediación, ya sean disputa o bajo rendimiento, el responsable del grupo de desarrollo intervendrá realizando aquellas acciones que considere necesaria. En caso de disputa si su resolución no satisface a ambas partes, el director del proyecto junto con el responsable del grupo y las dos partes de la disputa se reunirán para tomar las decisiones y resoluciones necesarias para finalizar la disputa.

El proyecto será almacenado en un repositorio central donde se llevará el control de versiones. Al finalizar el proyecto se otorgará al cliente el repositorio, el codigo fuente y el control de la aplicación ya desplegada. Al final de cada semana se estudiará el progreso del proyecto según las pautas marcadas en el diagrama de Gantt, pudiendo revisar y actualizar dicho diagrama.

\subsubsection{Procesos técnicos}

La generación de documentación del sistema se llevará a cabo por el propio equipo de desarrollo durante la implementación de la aplicación. Esta documentación será almacenada y actualizada dentro del repositorio central. Para generar la documentación general se seguirá el estandar UML utilizando la herramienta StarUML en su versión de prueba. En el caso de las diferentes clases y paquetes que componen el sistema se utilizarán las herrramientas disponibles para generar la documentación automáticamente como Javadoc y JSdoc.

Para el desarrollo de los diferentes paquetes y componentes que componen el sistema se seguirá la metodología de programación en parejas. Donde dos personas construyen una misma clase, uno escribe el código fuente mientras que el otro supervisa la corrección del código. A lo largo del proyecto cada equipo de desarrollo utilizará diferentes herramientas de desarrollo para falicitar la implementación del proyecto, la gestión de versiones y la compilación o interpretación de cada lenguaje utilizado. En general se utlizarán los IDEs de JetBrains. IntelliJ IDEA para Java, WebStorm para JS, DataGrip para SQL y modelado de la base de datos.

El despliegue de la aplicación se llevará a cabo utilizando los servicios de Amazon AWS que facilitan un despliegue rápido y sencillo.

\subsection{Planes}
\label{pl}
\subsubsection{Plan de gestión de configuraciones}
\begin{itemize}
		% \item \textbf{Convenciones de nombres (documentos) y estándares de código.}
    \item La documentación estará centralizada en un repositorio online. Se dividirá la documentación en varios archivos fuente de \LaTeX~ en función de las secciones del documento. De este modo existirán los siguientes archivos:
		\begin{itemize}
			\item \textit{main.tex:} Contiene las macros de la documentación, la introducción y la estructura del resto del documento.
			\item \textit{2-organización.tex:} Contenido de la Sección \textit{Organización del Proyecto}.
			\item \textit{3-gestion.tex:} Contenido de la Sección \textit{Plan de Gestión del Proyecto}.
			\item \textit{4-analisis-diseno.tex:} Contenido de la Sección \textit{Análisis y Diseño del Sistema}.
			\item \textit{5-memoria.tex:} Contenido de la Sección \textit{Memoria del Proyecto}.
			\item \textit{data.bib:} Base de Datos con la bibliografía usada.
		\end{itemize}
		Las imágenes, figuras y diagramas que se utilicen desde esos archivos estarán guardadas en la carpeta \textit{figuras} con un nombre descriptivo. Además existirá una versión compilada (PDF) de la última versión de los documentos así como los documentos anteriormente entregados con el nombre \textit{main-aaaa-mm-dd.pdf}.

		La \textit{Propuesta Técnica y Económica} estará disponible en formato Word en el fichero \textit{propuesta.docx} al igual que su documento compilado (PDF) y los documentos anteriormente entregados con el nombre \textit{propuesta-aaaa-mm-dd.pdf}

		Las actas de reuniones estarán en la carpeta \textit{actas} y sus nombres seguirán el patrón \textit{acta-aaaa-mm-dd}.

		La contabilidad de las horas de trabajo empleadas por cada miembro del equipos se encontrará en un fichero de cálculo llamado \textit{contabilidad-horas}.

		\item Se utilizarán las guías de estilo de Google para Java, Javascript, HTML y CSS \cite{googleguide}. En caso de no haber una guía de estilo disponible para un lenguaje adicional, se consensuará una para todos los equipos del proyecto.

    % \item\textbf{ Responsable o responsables de las distintas actividades (puesta en marcha, apoyo al equipo, revisión de commits, copias de seguridad, control de las versiones entregadas a cliente...).}
    \item{Se realizarán commits frecuentes siempre que el código compile. Los commits deben representar avances lógicos y atómicos de trabajo. Al programar en parejas, la revisión de los commits por parte de los compañeros de equipo (en parejas o tríos) se produce de forma natural. Los responsables de equipo serán encargados de realizar una copia de seguridad semanal off-site de su parte del repositorio mientras que el director del proyecto la realizará del repositorio completo. El director también será el responsable directo del control de las versiones entregadas al cliente. }
    % \item \textbf{Recursos: repositorios de control de versiones (cuáles, cuántos, permisos de acceso a los mismos) y sistema de gestión de incidencias.}
    \item{El equipo contará con un repositorio central en Github y se utilizará el gestor de incidencias integrado.}
    % \item \textbf{Procedimiento para realizar cambios al código fuente y los documentos técnicos: workflow de control de versiones utilizado, cuándo/cómo se permiten realizar commits al repositorio compartido, si tienen que ser aceptados por alguien previamente o no, qué hay que anotar en el sistema de gestión de incidencias, quién decide el estado de las incidencias, en qué estados puede estar una incidencia etc.}
		\item Los equipos realizarán su trabajo sobre una rama basada en la rama central del repositorio (rama del equipo). Los miembros de los equipos realizarán los commits sobre la rama del equipo (aunque podrán tener ramas auxiliares basadas en la rama del equipo). El responsable de cada equipo será quien podrá incluir los avances hechos sobre una rama de equipo en la rama central del repositorio.
    \item{Sólo se podrán realizar commits al repositorio central si el código compila y la parte modificada no afecta a la interfaz con elementos desarrollados por otros equipos. Si la interfaz se modifica, debe notificarse y recibir aprobación del resto de equipos afectados para realizar el commit. Las incidencias serán redirigidas a los equipos responsables de la parte de aplicación afectada y el responsable del equipo será el que tenga la potestad de cerrarla definitivamente.}
\end{itemize}
\subsubsection{Plan de construcción y despliegue del software}
\begin{itemize}
\item { La construcción del software se desarrollará con la utilización del framework Intelij para el desarrollo el software. Para llevar un control de versiones se creará un repositorio central en GitHub en el que se subirán las versiones a la rama master. Para verificar el correcto funcionamiento se ejecutarán sobre el software pruebas manuales y capacidad. Finalmente el despliegue se llevará a cabo de forma manual.}
\begin{figure}[H]
\centering
\includegraphics[scale = 0.4]{figuras/despliegue.jpg}
\caption{Diagrama de despliegue}
\label{fig:diagramaDespliegue}
\end{figure}

    \item{ El software se desplegará en tres niveles. El cliente, utilizando un navegador Chrome, interactuará con el software desplegado en el servidor web, a través de una conexión https. El servidor web mantendrá el puerto 443 abierto para permitir las conexiones https efectuadas por los clientes. El software del servidor web se desplegará en Amazon AWS, en una instancia de EC2, y este interactúa con la base de datos desplegada en Amazon Aurora para la creación de partidas o la modificación de los datos de un jugador. La comunicación entre el servidor web y la base de datos se desarrollará en una intranet utilizando la API JDBC que permite la ejecución de operaciones sobre bases de datos.}
\end{itemize}
\subsubsection{Plan de aseguramiento de la calidad}
Uno de los pilares del proyecto es el control de la calidad del software. Para ello el equipo intentará automatizar las tareas a este respecto todo lo posible y apoyarse en los siguientes pilares:
\begin{itemize}
\item{Para garantizar el correcto funcionamiento de los paquetes, clases y funciones generados se relizarán diferentes test unitarios de caja blanca y caja negra. Para las pruebas unitarias en Java y Javascript, se utilizarán JUnit y unit.js respectivamente. Estas pruebas deberán ser satisfactorias antes de cada commit para, adicionalmente, dar un mínimo de garantías de funcionamiento correcto del código que se almacena en el repositorio.}
\item{Para la integración de los diferentes modulos entre sí, se realizarán test de integración.}
\item{Como guías de estilo, se utilizarán las de Google para Java, Javascript, HTML y CSS. En el caso de que a lo largo del desarrollo se introduzca algún lenguaje nuevo, los responsables de equipo y el director consensuarán el uso de una guía de uso concreta. En caso de no existir una, se creará un documento con directivas importantes a seguir al utilizar ese lenguaje.}
\item{Para representar y especificar el sistema tanto dentro como fuera de la organización, se utilizará el estándar UML, agilizando y concretando la comunicación entre equipos, evitando errores causados por una mala comprensión de la arquitectura del sistema.}
\item{Se programará por parejas las partes críticas de la lógica del juego y de la aplicación para reducir el número de errores y mejorar la calidad del código en general.}
\item{Cada 30 días se realizará una revisión de requisitos de la aplicación en la que se especificará los requisitos cumplidos y los pendientes.}
% \item  \textbf{Estándares de código y otros (se pueden definir guías para la documentación de diseño y otros documentos del proyecto).}
%     \item \textbf{ Actividades de control de calidad del código que se realizarán: revisiones de código por pares, revisiones de requisitos o diagramas UML por pares, tipos de tests automáticos o manuales que se llevarán a cabo.}
\end{itemize}
\subsubsection{Calendario del proyecto y división del trabajo}
	\begin{figure}[H]
		\hspace{-3cm}
		\includegraphics[scale=0.8]{figuras/gantt.pdf}
		\caption{Diagrama de Gantt}
	\end{figure}
% \begin{itemize}
%     \item\textbf{ Diagrama de Gantt que recoja las tareas a realizar. Tened en cuenta que trabajáis con dos iteraciones y por tanto que hay una entrega intermedia y una final, y reflejarlo en este diagrama. Tened en cuenta que es normal que lo tengáis que actualizar conforme avance el proyecto (cuándo y cómo establezcáis en la sección 3.1.2).}
%      \item \textbf{Debe quedar claro qué requisitos van a estar completados en la primera iteración y cuáles en la segunda. Es posible que para la primera iteración no se planifique completar ningún requisito, pero en ese caso tiene que planificarse qué se hará y que faltará por hacer para cada requisito.}
%     \item \textbf{División del trabajo en partes (los módulos del software a desarrollar, pero también  la documentación, el diseño gráfico, instalaciones o despliegues, pruebas manuales etc.) y reparto de los mismos entre el equipo de desarrollo, al menos a alto nivel (el reparto de labores concretas en el día a día no se detalla aquí, pero hay que explicar bajo qué criterios y quién/cómo se hace en la sección 3.1.2). Debe haber una correspondencia con las tareas que aparecen en el diagrama de Gantt (que no necesariamente tiene que ser una relación 1 a 1).}
%      \item \textbf{Verificar que esta división del trabajo cubre todos los requisitos}
% \end{itemize}

Como se observa en el diagrama, el proyecto está dividido en dos iteraciones, con sus correspondientes demostraciones al cliente del avance del proyecto. La primera iteración finaliza la semana del 9 de abril y la segunda, que se corresponde con la entrega final, el 1 de junio.\\

%TODO: terminar esta sección cuando marius acabe los requisitos (cambiando los numeros o quizas hablando de bloques de requisitos)
\textbf{Primera iteración}

En la primera iteración se presentará una partida funcional individual, así como la página web con las vistas de las pantallas de login y perfil de usuario. También estará disponible el modo espectador. La inteligencia artificial se habrá comenzado a diseñar pero no será todavía funcional.

En concreto los requisitos funcionales 2-6, 9-10 y los requisitos no funcionales 1-2 estarán cubiertos completamente. Además, el requisito funcional 1 estará cubierto en cuanto a las partidas individuales, pero quedará la implementación de las partidas por parejas. El requisito funcional 13 estará cubierto parcialmente, ya que los turnos tendrán un periodo de tiempo establecido, acabando la partida si el jugador no realiza movimiento, pero el sistema de puntuaciones no estará implementado por lo que no habrá penalización. De la misma forma, relativo a los requisitos funcionales 14 y 17, se podrá abandonar la partida manualmente pero todavía no habrá penalización de puntuaciones. Finalmente, el requisito funcional 24 correspondiente al desarrollo de la inteligencia artificial, quedará cubierto solo parcialmente, en lo que respecta a análisis y representación del problema pero no la implementación e integración con el resto del sistema.\\

\textbf{Segunda iteración}

En la segunda iteración o entrega final, se presentará al cliente el sistema con todas las características especificadas totalmente funcionales. Se añadirán a las funcionalidades de la primera iteración todas las correspondientes a las puntuaciones, las ligas y los torneos (así como su programación automática), el sistema de matchmaking, la tienda, el panel de administración y la inteligencia artificial.

En concreto, se cubrirán por completo los requisitos funcionales 1, 7-8, 11-24 y los no funcionales 3-7.


\subsubsection*{División del trabajo}
\label{repartotrabajo}
A continuación se detalla una división del proyecto en bloques, con el correspondiente equipo o equipos de los descritos en la sección \ref{organiz} que los llevarán a cabo. Además, se incluye la lista de requisitos (especificados en el apartado \ref{requisitos}) que quedan cubiertos en cada uno de estos bloques para garantizar que se satisfacen todos ellos.
%TODO: terminar con los requisitos

\begin{table}[H]
\label{divisionTrabajo}
\hspace{-0.8cm}
\begin{tabular}{|l|c|l|}
\hline
\multicolumn{1}{|c|}{\textbf{Bloque}}                          & \textbf{Equipo} & \multicolumn{1}{c|}{\textbf{Requisitos}}                                                                       \\ \hline
Desarrollo de la interfaz del guiñote                 & 1      & \begin{tabular}[c]{@{}l@{}}\small{RF: 1,6}\\ \small{RNF: 2}\end{tabular}                                              \\ \hline
Desarrollo de la lógica de juego del guiñote          & 4      & \begin{tabular}[c]{@{}l@{}}\small{RF: 1}\\ \small{RNF: -}\end{tabular}                                                \\ \hline
Diseño e implementación de vistas web                 & 1      & \begin{tabular}[c]{@{}l@{}}\small{RF: 2,4,6,7,8,9,11,15,18,19,20,21}\\ \small{RNF: 2}\end{tabular}                    \\ \hline
Implementación de la web dinámica                     & 4      & \begin{tabular}[c]{@{}l@{}}\small{RF: 1,6,12,13,14,16,17}\\ \small{RNF: 3,5}\end{tabular}                             \\ \hline
Diseño e implementación de las comunicaciones         & 1      & \begin{tabular}[c]{@{}l@{}}\small{RF: 1,3,4,12}\\ \small{RNF: 1,2}\end{tabular}                                       \\ \hline
Desarrollo de la base de datos                        & 2      & \begin{tabular}[c]{@{}l@{}} \small{RF: 2,4,7,9,11,13,14,15,16,17,18,19,20,22,23}\\ \small{RNF: 1,3,4,5,6}\end{tabular} \\ \hline
Diseño e implementación de la Inteligencia Artificial & 3      & \begin{tabular}[c]{@{}l@{}}\small{RF: 24}\\ \small{RNF: -}\end{tabular}                                               \\ \hline
Despliegue                                            & 2      & \begin{tabular}[c]{@{}l@{}}\small{RF: 10}\\ \small{RNF: 1}\end{tabular}                                               \\ \hline
\end{tabular}
\caption{Tabla de división del trabajo}
\end{table}
