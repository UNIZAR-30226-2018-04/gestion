\documentclass[]{article}
\usepackage[spanish]{babel}
\usepackage[utf8]{inputenc}
\usepackage{graphicx}
\usepackage{geometry}
\usepackage{textpos}
\usepackage{fancyhdr}
\usepackage{float}
\usepackage{array}
\usepackage{multirow}
\usepackage{listings}
\usepackage{tikz}
\usetikzlibrary{arrows,shapes,positioning,shadows,trees}


\usepackage[colorlinks=true,linkcolor=black,anchorcolor=black,citecolor=black,filecolor=black,menucolor=black,runcolor=black,urlcolor=black]{hyperref}
\pagestyle{fancy}
\setlength{\headheight}{40pt}
\renewcommand{\headrulewidth}{1pt}
\renewcommand{\footrulewidth}{1pt}

\lhead{\leftmark}
\rhead{\thepage}
\cfoot{}
\lfoot{Plan de Gestión, Análisis, Diseño Y Memoria del Proyecto}

\begin{document}
\begin{titlepage}
	\begin{textblock*}{100mm}(.75\textwidth,-1.5cm)
	Equipo: \textsc{Margaret Hamilton}
    \end{textblock*}
    \begin{textblock*}{100mm}(.66\textwidth,-1.15cm)
    \hspace{-3cm}Contacto:  Julia   716185@unizar.es / Sergio 721057@unizar.es
	\end{textblock*}
    \vfill
	\centering \includegraphics[scale=0.3]{figuras/margaret.png}\\[3em]
    \centering \huge \textsc{Proyecto Sota y Rey}\\[1em]
    \centering \large \textsc{Plan de gestión, análisis, diseño y memoria del proyecto}\\[3em]
    \centering \today
\end{titlepage}

\tableofcontents
\newpage


\section{Introducción}

El presente documento recoge el plan de gestión, análisis, diseño y memoria del proyecto de implementación de una aplicación web para jugar partidas de Guiñote online. La aplicación permite jugar a este conocido juego de cartas contra amigos o adversarios desconocidos de su mismo nivel ya sea en uno contra uno o por parejas. Además, si se prefiere, la aplicación ofrece una Inteligencia Artificial a la que poder retar para entrenarse sin necesidad de un contrincante real.\\

La aplicación no se limita a ser una plataforma para jugar partidas de Guiñote, sino que añade componentes de red social y funcionalidades para aumentar el atractivo del juego. El sistema utiliza un registro de perfil de usuario en el que se almacena un historial completo de cada jugador. Además, existe la posibilidad de visualizar partidas que se están jugando en ese momento en modo espectador. Las funcionalidades que incrementan la jugabilidad de la aplicación se basan en añadir competitividad. Existe un sistema de ligas al que se pertenece en función de las habilidades del usuario, medidas según el ratio de partidas ganadas y perdidas, y un sistema de torneos de eliminación directa a los que los jugadores pueden inscribirse. Como recompensa al desempeño realizado en los torneos y ligas existe una divisa virtual que los jugadores ganan y que se puede utilizar para comprar items en la tienda del juego, que les permiten personalizar sus partidas.\\

El sistema es accesible desde el navegador (al menos Google Chrome 64). Para que se pueda jugar desde distintos dispositivos la aplicación tiene una interfaz responsive, adaptándose al tamaño de la pantalla desde la que se juega. Este cambio de dispositivo puede hacerse incluso durante una partida en curso, siempre que no se agote el tiempo de turno correspondiente. \\

El sistema está alojado en Amazon AWS como un servidor ejecutado en una instancia virtual de EC2 mientras que todo lo referente al almacenamiento de información (datos jugadores, partidas, elementos de la tienda…) está almacenado en Amazon Aurora. \\

El desarrollo del proyecto está dividido en dos fases, lo que permite que pueda ser validado y contrastado con el cliente. La primera fase incluye un prototipo funcional de la aplicación que permite jugar partidas individuales, o en modo espectador e incluye las pantallas de login y perfil de usuario, aunque todavía no tenga implementada una Inteligencia Artificial funcional y competente. Esta versión del programa es mostrada al cliente el 9 de Abril de 2018.

Como resultado final del proyecto el cliente recibe la aplicación desplegada en AWS, el código fuente y el resultado de la compilación para la plataforma objetivo. Además, el cliente recibe un manual técnico completo y una memoria resumen del proceso de desarrollo. La fecha de entrega del producto terminado es el 1 de Junio de 2018.\\

\clearpage

\subsection{Estructura del documento}
En la sección \ref{organiz} se describe la organización del equipo de trabajo, incluyendo un organigrama de la empresa. A continuación, en la sección \ref{planes} se detalla el plan de gestión del proyecto. Esta sección incluye una primera parte relativa a cómo se llevarán a cabo las distintas tareas a lo largo del proyecto (subsección \ref{procesos}) y una segunda referente a los planes (subsección \ref{pl}), en la que se detallan el plan de gestión de configuraciones, el de despliegue, aseguramiento de la calidad y el plan de trabajo especificado en un diagrama de Gantt. Finalmente, se incluye en la sección \ref{analisis} una descripción detallada de los requisitos funcionales y no funcionales del sistema, así como el diseño del sistema, detallado mediante un diagrama de componentes.

\section{Organización del proyecto}
\label{organiz}

Para incentivar el trabajo en paralelo en proyecto, se han creado diferentes grupos de trabajo identificados por las tareas que conllevan:

\begin{figure}[h]
	\centering \includegraphics[scale=0.6]{figuras/organigrama.png}
\end{figure}


\begin{enumerate}
\item Desarrollo del Frontend y Middleware: Incluye el desarrollo de la interfaz y las vistas de la aplicación, además de la gestión de la comunicación entre los jugadores y servidor durante el desarrollo la partida.
	\begin{itemize}
    \item Carlos Marañés \textbf{(responsable)}.
		\item Javier Corbalán.
    	\item Ignacio Bitrián.
	\end{itemize}
\newpage
\item Despliegue e implementación de la base de datos: Incluye el análisis y diseño de la base de datos, gestión de los datos de carácter personal de los usuarios acorde con la LOPD\footnote{\href{http://www.agpd.es/portalwebAGPD/canaldocumentacion/informes_juridicos/reglamento_lopd/index-ides-idphp.php}{Ley Orgánica de Protección de Datos}} , despliegue de la aplicación y mantenimiento.
	\begin{itemize}
    	\item Sergio Izquierdo \textbf{(responsable)}.
        \item Julia Guerrero.
    \end{itemize}
\item Desarrollo de la inteligencia artificial: Incluye el análisis del juego, diseño e implementación de un agente de inteligencia artificial.
	\begin{itemize}
    	\item Victor Soria \textbf{(responsable)}.
        \item Marius Crisan.
    \end{itemize}
\item Desarrollo del Backend: Implementación de los servicios web, tratamiento de peticiónes, generación de páginas dinámicas y la lógica del juego.
	\begin{itemize}
   		\item Marius Crisan \textbf{(responsable)}.
    	\item Ignacio Bitrian.
        \item Víctor Soria.
    \end{itemize}
\end{enumerate}

El rol de director del proyecto corre a cargo de Javier Corbalán. La elección del director se ha llevado a cabo através de una votación de mayoría simple.

\clearpage
\section{Plan de gestión del proyecto}
\label{planes}
\subsection{Procesos}
\label{procesos}

El proyecto está dividido en dos iteraciones, cada una de ellas posee cinco fases: requisitos, análisis, diseño, implementacion y pruebas. En la primara iteración se va a desarrollar el funcionamiento de una partida tanto individual como por parejas y la página web con las pantallas de login y perfil de usuario. En la segunda iteración se va a desarrollar el resto de funcionalidades del sistema, que incluyen la inteligencia artificial, torneos, tienda, entre otros.  Cada fase de cada iteración tendrá una duración máxima de una semana, excepto la fase de recogida de requisitos que será de duración inferior y que solo se llevará acabo al inicio del proyecto.
A lo largo del proyecto el equipo se dividirá en los grupos de desarrollo: bases de datos y despliegue, backend, frontend y middelware e IA. Donde la participación será flexible pero que mantendrá de forma fija a un responsable.

\subsubsection{Procesos de inicio del proyecto}
\label{inicio}

Al iniciar el proyecto, cada grupo de desarrollo realizará y ejecutará un plan de adquisición de aquellos recursos necesarios para el desarrollo de su objetivo. Dichos recursos se limitarán a aquellos que figuran en la propuesta económica, a excepción de recursos extra que considere cada responsable de grupo. Estos recursos extra, si poseen un coste economico, deberán ser presentados y justificados al director del proyecto, el cual aprobará o rechazará.

En la propuesta económica figura únicamente la contratación de un servicio cloud en Amazon AWS y la adquisición de un domio web para acceder a la aplicación. El equipo de bases de datos y despliegue completará el registro y configuración de una cuenta en Amazon y realizará un estudio para finalmente adquirir un dominio web. Al finalizar el proyecto estos recursos serán traspasados al cliente. El resto de recursos utilizados qué no aparecen en la propuesta técnica y económica, poseen una licencia de uso libre y por lo tanto no suponen un coste a la empresa, por lo que no aparecen justificados en la propuesta económica. Estos recursos incluyen diferentes IDEs para el desarrollo de la aplicación y un certificado TLS gratuito obtenido através del servicio "Let's Encrypt". Cada equipo gestionará la adquisición de sus recursos.

Dado que que el proyecto integra un gran número de tecnologías y capas que conforman la arquitectura web, es necesario realizar un estudio de viabilidad de las tecnologías escogidas antes de realizar la adquisición del software de desarrollo, este estudio requerirá de 5 horas de trabajo entre todos los grupos.  En un principio el desarrollo del proyecto se realizará sobre JavaEE, JSP, WebSockets, JavaScript, Bootstrap y MySQL. Dado que una parte del equipo de desarrollo posee cierta experiencia en el desarrollo web, no tanto en la utilización de estas herramientas, es necesario que se lleve a cabo un proceso de aprendizaje inicial. Este proceso requerirá número mínimo de horas que peprmitirá al equipo comenzar con el desarrollo del proyecto. No obstante a lo largo del proyecto la formación y la adquisición de experiencia serán fundamentales.\\
En aquellas tecnologías donde el equipo de desarrollo no posee ninguna experiencia, como WebSockets, JavaScript , JSP,  entre otros, se requerirá un esfuerzo extra por parte del equipo para auto-formarse mediante la lectura de libros, tutoriales y la documentación pertinente.
\begin{itemize}
	\item Tanto Carlos Marañés como Javier Corbalán se formarán en Javascript, Websockets y Phaser.io a través del uso de ejemplos y documentación en línea.
	\item Julia Guerrero y Sergio Izquierdo se formarán en HTML/CSS y JSP con documentación en línea y  con la ayuda de los miembros del equipo que tienen experiencia en esa tecnología.
    \item El equipo entero se formará en la dinámica de trabajo con AWS gracias a la experiencia de Sergio Izquierdo y tutoriales en línea.
\end{itemize}

\subsubsection{Procesos de ejecución y control del proyecto}

Una de las funciones más importantes del director del proyecto es gestionar la comunicación dentro del equipo. Para garantizar que esta máxima se cumple, el director tendrá la capacidad de convocar reuniones que incluyan a todo el equipo o solo a los responsables de cada grupo de desarrollo. En cada reunión se realizará un acta que recoja todos los aspectos y decisiones importantes acaecidos en la reunión. Las decisiones tomadas en estas reuniones deberán trasladarse a la documentación del proyecto y finalmente a la implementación.
\\\\
Los grupos de desarrollo se coordinarán de forma autónoma para no sobrecargar la figura de director, para ello existirá la figura de responsable, ya avanzada anteriormente. Si el equipo de trabajo sigue un correcto funcionamiento el director solo deberá reunirse con los responsables de cada grupo, sin embargo en ocasiones extraordinarias podrá reunirse con todo el grupo para tomar decisiones que engloben a todo el proyecto, o para corregir posibles malfuncionamientos. Los objetivos globales de cada grupo serán supervisados semana a semana por el director del proyecto. Mientras que dentro de cada grupo el responsable asignará día a día las tareas necesarias para cumplir con los objetivos. Los objetivos que semanalmente cada grupo de desarrollo se marca deben ser sencillos, concretos y trazables, de forma que permitan  medir semanalmente el progreso del proyecto. Si una semana un grupo de desarrollo no cumple con las tareas asignadas el responsable puede reforzar el grupo con nuevos miembros y enfocando las tareas de forma diferente para completar esos objetivos antes de la semana siguiente.

En aquellas situaciones donde se requiera mediación, ya sean disputa o bajo rendimiento, el responsable del grupo de desarrollo intervendrá realizando aquellas acciones que considere necesaria. En caso de disputa si su resolución no satisface a ambas partes, el director del proyecto junto con el responsable del grupo y las dos partes de la disputa se reunirán para tomar las decisiones y resoluciones necesarias para finalizar la disputa.

El proyecto será almacenado en un repositorio central donde se llevará el control de versiones. Al finalizar el proyecto se otorgará al cliente el repositorio, el codigo fuente y el control de la aplicación ya desplegada. Al final de cada semana se estudiará el progreso del proyecto según las pautas marcadas en el diagrama de Gantt, pudiendo revisar y actualizar dicho diagrama.

\subsubsection{Procesos técnicos}

La generación de documentación del sistema se llevará a cabo por el propio equipo de desarrollo durante la implementación de la aplicación. Esta documentación será almacenada y actualizada dentro del repositorio central. Para generar la documentación general se seguirá el estandar UML utilizando la herramienta StarUML en su versión de prueba. En el caso de las diferentes clases y paquetes que componen el sistema se utilizarán las herrramientas disponibles para generar la documentación automáticamente como Javadoc y JSdoc.

Para el desarrollo de los diferentes paquetes y componentes que componen el sistema se seguirá la metodología de programación en parejas. Donde dos personas construyen una misma clase, uno escribe el código fuente mientras que el otro supervisa la corrección del código. A lo largo del proyecto cada equipo de desarrollo utilizará diferentes herramientas de desarrollo para falicitar la implementación del proyecto, la gestión de versiones y la compilación o interpretación de cada lenguaje utilizado. En general se utlizarán los IDEs de JetBrains. IntelliJ IDEA para Java, WebStorm para JS, DataGrip para SQL y modelado de la base de datos.

El despliegue de la aplicación se llevará a cabo utilizando los servicios de Amazon AWS que facilitan un despliegue rápido y sencillo.

\subsection{Planes}
\label{pl}
\subsubsection{Plan de gestión de configuraciones}
\begin{itemize}
		% \item \textbf{Convenciones de nombres (documentos) y estándares de código.}
    \item La documentación estará centralizada en un repositorio online. Se dividirá la documentación en varios archivos fuente de \LaTeX~ en función de las secciones del documento. De este modo existirán los siguientes archivos:
		\begin{itemize}
			\item \textit{main.tex:} Contiene las macros de la documentación, la introducción y la estructura del resto del documento.
			\item \textit{2-organización.tex:} Contenido de la Sección \textit{Organización del Proyecto}.
			\item \textit{3-gestion.tex:} Contenido de la Sección \textit{Plan de Gestión del Proyecto}.
			\item \textit{4-analisis-diseno.tex:} Contenido de la Sección \textit{Análisis y Diseño del Sistema}.
			\item \textit{5-memoria.tex:} Contenido de la Sección \textit{Memoria del Proyecto}.
			\item \textit{data.bib:} Base de Datos con la bibliografía usada.
		\end{itemize}
		Las imágenes, figuras y diagramas que se utilicen desde esos archivos estarán guardadas en la carpeta \textit{figuras} con un nombre descriptivo. Además existirá una versión compilada (PDF) de la última versión de los documentos así como los documentos anteriormente entregados con el nombre \textit{main-aaaa-mm-dd.pdf}.

		La \textit{Propuesta Técnica y Económica} estará disponible en formato Word en el fichero \textit{propuesta.docx} al igual que su documento compilado (PDF) y los documentos anteriormente entregados con el nombre \textit{propuesta-aaaa-mm-dd.pdf}

		Las actas de reuniones estarán en la carpeta \textit{actas} y sus nombres seguirán el patrón \textit{acta-aaaa-mm-dd}.

		La contabilidad de las horas de trabajo empleadas por cada miembro del equipos se encontrará en un fichero de cálculo llamado \textit{contabilidad-horas}.

		\item Se utilizarán las guías de estilo de Google para Java, Javascript, HTML y CSS \cite{googleguide}. En caso de no haber una guía de estilo disponible para un lenguaje adicional, se consensuará una para todos los equipos del proyecto.

    % \item\textbf{ Responsable o responsables de las distintas actividades (puesta en marcha, apoyo al equipo, revisión de commits, copias de seguridad, control de las versiones entregadas a cliente...).}
    \item{Se realizarán commits frecuentes siempre que el código compile. Los commits deben representar avances lógicos y atómicos de trabajo. Al programar en parejas, la revisión de los commits por parte de los compañeros de equipo (en parejas o tríos) se produce de forma natural. Los responsables de equipo serán encargados de realizar una copia de seguridad semanal off-site de su parte del repositorio mientras que el director del proyecto la realizará del repositorio completo. El director también será el responsable directo del control de las versiones entregadas al cliente. }
    % \item \textbf{Recursos: repositorios de control de versiones (cuáles, cuántos, permisos de acceso a los mismos) y sistema de gestión de incidencias.}
    \item{El equipo contará con un repositorio central en Github y se utilizará el gestor de incidencias integrado.}
    % \item \textbf{Procedimiento para realizar cambios al código fuente y los documentos técnicos: workflow de control de versiones utilizado, cuándo/cómo se permiten realizar commits al repositorio compartido, si tienen que ser aceptados por alguien previamente o no, qué hay que anotar en el sistema de gestión de incidencias, quién decide el estado de las incidencias, en qué estados puede estar una incidencia etc.}
		\item Los equipos realizarán su trabajo sobre una rama basada en la rama central del repositorio (rama del equipo). Los miembros de los equipos realizarán los commits sobre la rama del equipo (aunque podrán tener ramas auxiliares basadas en la rama del equipo). El responsable de cada equipo será quien podrá incluir los avances hechos sobre una rama de equipo en la rama central del repositorio.
    \item{Sólo se podrán realizar commits al repositorio central si el código compila y la parte modificada no afecta a la interfaz con elementos desarrollados por otros equipos. Si la interfaz se modifica, debe notificarse y recibir aprobación del resto de equipos afectados para realizar el commit. Las incidencias serán redirigidas a los equipos responsables de la parte de aplicación afectada y el responsable del equipo será el que tenga la potestad de cerrarla definitivamente.}
\end{itemize}
\subsubsection{Plan de construcción y despliegue del software}
\begin{itemize}
\item { La construcción del software se desarrollará con la utilización del framework Intelij para el desarrollo el software. Para llevar un control de versiones se creará un repositorio central en GitHub en el que se subirán las versiones a la rama master. Para verificar el correcto funcionamiento se ejecutarán sobre el software pruebas manuales y capacidad. Finalmente el despliegue se llevará a cabo de forma manual.}
\begin{figure}[H]
\centering
\includegraphics[scale = 0.4]{figuras/despliegue.jpg}
\caption{Diagrama de despliegue}
\label{fig:diagramaDespliegue}
\end{figure}

    \item{ El software se desplegará en tres niveles. El cliente, utilizando un navegador Chrome, interactuará con el software desplegado en el servidor web, a través de una conexión https. El servidor web mantendrá el puerto 443 abierto para permitir las conexiones https efectuadas por los clientes. El software del servidor web se desplegará en Amazon AWS, en una instancia de EC2, y este interactúa con la base de datos desplegada en Amazon Aurora para la creación de partidas o la modificación de los datos de un jugador. La comunicación entre el servidor web y la base de datos se desarrollará en una intranet utilizando la API JDBC que permite la ejecución de operaciones sobre bases de datos.}
\end{itemize}
\subsubsection{Plan de aseguramiento de la calidad}
Uno de los pilares del proyecto es el control de la calidad del software. Para ello el equipo intentará automatizar las tareas a este respecto todo lo posible y apoyarse en los siguientes pilares:
\begin{itemize}
\item{Para garantizar el correcto funcionamiento de los paquetes, clases y funciones generados se relizarán diferentes test unitarios de caja blanca y caja negra. Para las pruebas unitarias en Java y Javascript, se utilizarán JUnit y unit.js respectivamente. Estas pruebas deberán ser satisfactorias antes de cada commit para, adicionalmente, dar un mínimo de garantías de funcionamiento correcto del código que se almacena en el repositorio.}
\item{Para la integración de los diferentes modulos entre sí, se realizarán test de integración.}
\item{Como guías de estilo, se utilizarán las de Google para Java, Javascript, HTML y CSS. En el caso de que a lo largo del desarrollo se introduzca algún lenguaje nuevo, los responsables de equipo y el director consensuarán el uso de una guía de uso concreta. En caso de no existir una, se creará un documento con directivas importantes a seguir al utilizar ese lenguaje.}
\item{Para representar y especificar el sistema tanto dentro como fuera de la organización, se utilizará el estándar UML, agilizando y concretando la comunicación entre equipos, evitando errores causados por una mala comprensión de la arquitectura del sistema.}
\item{Se programará por parejas las partes críticas de la lógica del juego y de la aplicación para reducir el número de errores y mejorar la calidad del código en general.}
\item{Cada 30 días se realizará una revisión de requisitos de la aplicación en la que se especificará los requisitos cumplidos y los pendientes.}
% \item  \textbf{Estándares de código y otros (se pueden definir guías para la documentación de diseño y otros documentos del proyecto).}
%     \item \textbf{ Actividades de control de calidad del código que se realizarán: revisiones de código por pares, revisiones de requisitos o diagramas UML por pares, tipos de tests automáticos o manuales que se llevarán a cabo.}
\end{itemize}
\subsubsection{Calendario del proyecto y división del trabajo}
	\begin{figure}[H]
		\hspace{-3cm}
		\includegraphics[scale=0.8]{figuras/gantt.pdf}
		\caption{Diagrama de Gantt}
	\end{figure}
% \begin{itemize}
%     \item\textbf{ Diagrama de Gantt que recoja las tareas a realizar. Tened en cuenta que trabajáis con dos iteraciones y por tanto que hay una entrega intermedia y una final, y reflejarlo en este diagrama. Tened en cuenta que es normal que lo tengáis que actualizar conforme avance el proyecto (cuándo y cómo establezcáis en la sección 3.1.2).}
%      \item \textbf{Debe quedar claro qué requisitos van a estar completados en la primera iteración y cuáles en la segunda. Es posible que para la primera iteración no se planifique completar ningún requisito, pero en ese caso tiene que planificarse qué se hará y que faltará por hacer para cada requisito.}
%     \item \textbf{División del trabajo en partes (los módulos del software a desarrollar, pero también  la documentación, el diseño gráfico, instalaciones o despliegues, pruebas manuales etc.) y reparto de los mismos entre el equipo de desarrollo, al menos a alto nivel (el reparto de labores concretas en el día a día no se detalla aquí, pero hay que explicar bajo qué criterios y quién/cómo se hace en la sección 3.1.2). Debe haber una correspondencia con las tareas que aparecen en el diagrama de Gantt (que no necesariamente tiene que ser una relación 1 a 1).}
%      \item \textbf{Verificar que esta división del trabajo cubre todos los requisitos}
% \end{itemize}

Como se observa en el diagrama, el proyecto está dividido en dos iteraciones, con sus correspondientes demostraciones al cliente del avance del proyecto. La primera iteración finaliza la semana del 9 de abril y la segunda, que se corresponde con la entrega final, el 1 de junio.\\

%TODO: terminar esta sección cuando marius acabe los requisitos (cambiando los numeros o quizas hablando de bloques de requisitos)
\textbf{Primera iteración}

En la primera iteración se presentará una partida funcional individual, así como la página web con las vistas de las pantallas de login y perfil de usuario. También estará disponible el modo espectador. La inteligencia artificial se habrá comenzado a diseñar pero no será todavía funcional.

En concreto los requisitos funcionales 2-6, 9-10 y los requisitos no funcionales 1-2 estarán cubiertos completamente. Además, el requisito funcional 1 estará cubierto en cuanto a las partidas individuales, pero quedará la implementación de las partidas por parejas. El requisito funcional 13 estará cubierto parcialmente, ya que los turnos tendrán un periodo de tiempo establecido, acabando la partida si el jugador no realiza movimiento, pero el sistema de puntuaciones no estará implementado por lo que no habrá penalización. De la misma forma, relativo a los requisitos funcionales 14 y 17, se podrá abandonar la partida manualmente pero todavía no habrá penalización de puntuaciones. Finalmente, el requisito funcional 24 correspondiente al desarrollo de la inteligencia artificial, quedará cubierto solo parcialmente, en lo que respecta a análisis y representación del problema pero no la implementación e integración con el resto del sistema.\\

\textbf{Segunda iteración}

En la segunda iteración o entrega final, se presentará al cliente el sistema con todas las características especificadas totalmente funcionales. Se añadirán a las funcionalidades de la primera iteración todas las correspondientes a las puntuaciones, las ligas y los torneos (así como su programación automática), el sistema de matchmaking, la tienda, el panel de administración y la inteligencia artificial.

En concreto, se cubrirán por completo los requisitos funcionales 1, 7-8, 11-24 y los no funcionales 3-7.


\subsubsection*{División del trabajo}
\label{repartotrabajo}
A continuación se detalla una división del proyecto en bloques, con el correspondiente equipo o equipos de los descritos en la sección \ref{organiz} que los llevarán a cabo. Además, se incluye la lista de requisitos (especificados en el apartado \ref{requisitos}) que quedan cubiertos en cada uno de estos bloques para garantizar que se satisfacen todos ellos.
%TODO: terminar con los requisitos

\begin{table}[H]
\label{divisionTrabajo}
\hspace{-0.8cm}
\begin{tabular}{|l|c|l|}
\hline
\multicolumn{1}{|c|}{\textbf{Bloque}}                          & \textbf{Equipo} & \multicolumn{1}{c|}{\textbf{Requisitos}}                                                                       \\ \hline
Desarrollo de la interfaz del guiñote                 & 1      & \begin{tabular}[c]{@{}l@{}}\small{RF: 1,6}\\ \small{RNF: 2}\end{tabular}                                              \\ \hline
Desarrollo de la lógica de juego del guiñote          & 4      & \begin{tabular}[c]{@{}l@{}}\small{RF: 1}\\ \small{RNF: -}\end{tabular}                                                \\ \hline
Diseño e implementación de vistas web                 & 1      & \begin{tabular}[c]{@{}l@{}}\small{RF: 2,4,6,7,8,9,11,15,18,19,20,21}\\ \small{RNF: 2}\end{tabular}                    \\ \hline
Implementación de la web dinámica                     & 4      & \begin{tabular}[c]{@{}l@{}}\small{RF: 1,6,12,13,14,16,17}\\ \small{RNF: 3,5}\end{tabular}                             \\ \hline
Diseño e implementación de las comunicaciones         & 1      & \begin{tabular}[c]{@{}l@{}}\small{RF: 1,3,4,12}\\ \small{RNF: 1,2}\end{tabular}                                       \\ \hline
Desarrollo de la base de datos                        & 2      & \begin{tabular}[c]{@{}l@{}} \small{RF: 2,4,7,9,11,13,14,15,16,17,18,19,20,22,23}\\ \small{RNF: 1,3,4,5,6}\end{tabular} \\ \hline
Diseño e implementación de la Inteligencia Artificial & 3      & \begin{tabular}[c]{@{}l@{}}\small{RF: 24}\\ \small{RNF: -}\end{tabular}                                               \\ \hline
Despliegue                                            & 2      & \begin{tabular}[c]{@{}l@{}}\small{RF: 10}\\ \small{RNF: 1}\end{tabular}                                               \\ \hline
\end{tabular}
\caption{Tabla de división del trabajo}
\end{table}

\clearpage
\section{Análisis y diseño del sistema}
\label{analisis}
\subsection{Análisis de requisitos}
\label{requisitos}

\begin{table}[H]
\centering
\begin{tabular}{|c|p{12cm}|}
\hline
\multicolumn{2}{|c|}{GUIÑOTE} \\ \hline
RF-1 &  El sistema permite a los usuarios jugar al guiñote en modo uno contra uno y dos contra dos. \\ \hline
RF-2 & El sistema almacena un historial de partidas jugadas por el usuario y permite visualizar estadísticas de las partidas jugadas. \\ \hline
RF-3 & El jugador, en mitad de una partida, puede desconectarse y   volver a conectarse desde cualquier dispositivo siempre que no sea su turno o, en caso de serlo, no agote su tiempo de turno. \\ \hline
RF-4 & El sistema requiere que los usuarios se registren con su correo electrónico o Facebook para poder acceder a los servicios que ofrece. \\ \hline
RF-5 & El jugador puede seleccionar si desea jugar una partida pública o privada. \\ \hline
RF-6 & Los usuarios pueden elegir una partida pública en curso y unirse a ella como espectadores. \\ \hline
RF-7 & El sistema cuenta con una divisa virtual que se consigue al iniciar sesión por primera vez, ganando torneos, partidas, ascensos a otra liga, etc. \\ \hline
RF-8 & El sistema consta de un panel de administración al cual se pueden conectar solamente los usuarios definidos como administradores con anterioridad. \\ \hline
RF-9 & El sistema permite al usuario administrar sus datos personales almacenados en el sistema: nombre de usuario, avatar, correo electrónico. \\ \hline
RF-10 & El usuario debe conectarse utilizando el navegador web Google Chrome para garantizar el correcto funcionamiento del sistema. \\ \hline
RF-24 & El usuario puede jugar contra un agente de inteligencia artificial, únicamente en el modo uno contra uno.  \\ \hline
RNF-1 & El sistema garantiza la seguridad de la información de los usuarios. \\ \hline
RNF-2 & El sistema permite la conexión desde diferentes dispositivos. La aplicación es responsive por lo que se muestra de forma diferente para cada uno de los diferentes tamaños de pantalla. \\ \hline
RNF-3 & El sistema tarda menos de 20 segundos en encontrar partida aleatoria en caso de que haya un número de jugadores suficientes esperando para iniciar partida con las mismas características. \\ \hline
\end{tabular}
\caption{Requisitos relacionados con la jugabilidad y el usuario}
\label{tabla-usuario}
\end{table}


\begin{table}[H]
\centering
\begin{tabular}{|c|p{12cm}|}
\hline
\multicolumn{2}{|c|}{LIGAS} \\ \hline
RF-11 & Los jugadores tienen asociada una puntuación que varía en función de las partidas ganadas o perdidas. Dependiendo de ésta, pertenecerán a una u otra liga. \\ \hline
RF-12 & El sistema primero intenta emparejar a los jugadores de la misma liga. Si no es posible, los empareja con los de la liga más cercana a la suya. \\ \hline
RF-13 & Si el usuario no realiza ninguna acción durante su turno, la partida se termina y él recibe una penalización de puntuación. El turno es un periodo de tiempo preestablecido. \\ \hline
RF-14 & El sistema permite que los jugadores abandonen manualmente una partida de liga con una penalización asociada a su puntuación. \\ \hline
RNF-4 & Para ascender a una liga superior el jugador debe haber ganado muchas partidas. \\ \hline
\end{tabular}
\caption{Requisitos de las ligas}
\label{tabla-ligas}
\end{table}

\begin{table}[H]
\centering
\begin{tabular}{|c|p{12cm}|}
\hline
\multicolumn{2}{|c|}{TORNEOS} \\ \hline
RF-15 & El sistema permite a los usuarios inscribirse y participar en torneos. Los torneos son competiciones con eliminatorias directas a una partida única, en las cuales se pasa a la siguiente ronda ganando la partida. \\ \hline
RF-16 & Por cada ronda del torneo ganada el jugador recibe una puntuación proporcional a la ronda del torneo siendo la mayor bonificación la del ganador del torneo. \\ \hline
RF-17 & Los jugadores que abandonen una partida de un torneo son descalificados con una penalización asociada a su puntuación. \\ \hline
RF-18 & El administrador puede programar la creación automática de torneos ya sean puntuales o periódicos, especificando un momento de inicio y unos premios determinados. \\ \hline
RF-19 & El administrador puede modificar y eliminar torneos que aún no estén en curso. \\ \hline
RNF-5 & La puntuación recibida en cada fase por ganar una partida es mucho mayor que la recibida en una partida de liga y el doble que la de la fase anterior del mismo torneo. La puntuación para el campeón y el subcampeón es mucho más grande que la de los otros participantes. \\ \hline
RNF-6 & Los torneos programados inicialmente son semanales. \\ \hline
\end{tabular}
\caption{Requisitos de los torneos}
\label{tabla-torneo}
\end{table}

\begin{table}[H]
\centering
\label{tabla-tienda}
\begin{tabular}{|c|p{12cm}|}
\hline
\multicolumn{2}{|c|}{TIENDA} \\ \hline
RF-20 & El sistema posee una tienda para personalizar el tablero de juego, las barajas y el avatar, que inicialmente consta de 3 barajas, 3 tableros y 20 avatares diferentes. \\ \hline
RF-21 & El administrador de la aplicación puede añadir artículos nuevos a la tienda y modificar el   precio de los existentes \\ \hline
RF-22 & Los artículos de la tienda se desbloquean en función de la liga más alta en la que ha participado el usuario en algún momento. \\ \hline
RF-23 & Los artículos se compran con la divisa virtual que el usuario tiene acumuladas. \\ \hline
RNF-7 & El valor en divisas virtuales de las barajas es más mayor que el de los tableros. El precio de los avatares varía para cada uno de ellos y para conseguirlos el jugador debe haber ganado muchas partidas. \\ \hline
\end{tabular}
\caption{Requisitos de la tienda}
\end{table}


\subsection{Diseño del sistema}
Se ha decidido implementar una aplicación web de 3 capas debido a la gran popularidad de los sistemas web hoy en día. Las aplicaciones web permiten a multitud de usuarios acceder al servicio desde diferentes dispositivos ya sean moviles, tablets, u ordenadores de sobremesa. Sin embargo el acceso desde diferentes tamaños de pantallas suponen un gran problema de accesibilidad. Por lo que es necesario realizar la aplicación de forma que se adapte a los diferentes tipos de pantallas. \\
\\
Se han barajado las alternativas de hacer una aplicación Android o una aplicación de escritorio, pero para evitar invertir gran parte del tiempo en formación para poder desarrollar software en estos tipos de dispositivos, lo que supone un incremento en el número de horas del presupuesto, se ha optado por una aqutitectura en la que los miembros del equipo tengan algo más de experiencia. Además el desarrollo de este tipo de aplicaciónes están limitados por el entorno en el que se ejecuta la aplicación, ya que aquellas aplicaciones que se desarrollen para android no podrán ejecutarse sobre IOS ni las aplicaciones de windows sobre linux.
\\
La aplicación web que se va a desarrollar sigue un despliegue en tres capas con un navegador web ligero. La distribución de las diferentes partes se puede apreciar en el diagrama de componentes de la figura \ref{fig:diagramaComponentes}

\begin{figure}[H]
\centering
\includegraphics[scale = 0.5]{figuras/componentes.png}
\caption{Diagrama de componentes donde se refleja la distribución del sistema}
\label{fig:diagramaComponentes}
\end{figure}

\subsubsection{Base de Datos}
El primer paso en el diseño de la base de datos fue plantear el esquema entidad relación como esquema conceptual del problema. Una vez diseñado el paso a un esquema relacional fue sencillo.

\begin{figure}[H]
\centering
\includegraphics[scale = 0.5]{figuras/base_datos/diagrama-conceptual.pdf}
\caption{Esquema Conceptual}
\label{fig:diagramaConceptual}
\end{figure}

Antes de llegar al esquema actual se plantearon otras posibilidades, primeramente por una falta de claridad entre los datos que debían ser persistentes y los que no. En un primer momento se creó una entidad Espera, débil respecto a un Usuario que representaba que un usuario estaba esperando para encontrar jugadores. Finalmente la entidad no existe y la información de una espera no es almacenada en la base de datos. Las partidas son introducidas en la base cuando se empieza una partida, para así poder tener la información de las partidas en curso. La parte con más dificultad es la relacionada con los torneos. Para representar los torneos existen las fases, entendidas como octavos, cuartos, semifinales... Una partida puede estar ligada a una fase, y la fase pertenece a un torneo. Finalmente los jugadores están relacionados con una fase para poder emparejar jugadores.\\

Para la comunicación del sistema con la base de datos se ha utilizado el patrón DAO (Objeto de Acceso a Datos). Para ello se han implementado una serie de objetos VO que representan la información guardada de forma persistente y unos objetos DAO que abstraen las operaciones con el JDBC mediante métodos de Java. Para mayor seguridad de los datos, todos los posibles usos de la base se realizan a través de una interfaz que proporciona los métodos necesarios además de ofrecer un \textit{pool} de conexiones para aumentar la velocidad de la interacción cuando haya múltiples usuarios.\\

\begin{figure}[H]
\centering
\includegraphics[scale = 0.5]{figuras/base_datos/Componentes.pdf}
\caption{Esquema de la estructura del acceso a datos}
\label{fig:componentesbases}
\end{figure}

Los clientes que usen la base de datos deben realizar todas las operaciones a través de la Interfaz de Datos. La Interfaz utiliza el pool manager para no tener que crear una nueva conexión con cada cliente y poder reutilizarlas. La Interfaz invocará a las clases DAO que utilizaran las conexiones para conectarse finalmente con la base. Las clases VO sirven para representar los objetos de la base en la comunicación entre las clases.\\

Dado el gran número de clases empleado en la implementación del modelo, su representación se ha realizado mediante diferentes diagramas de clases enfocados en las diferentes funcionalidades de la base .
\begin{figure}[H]
\centering
\includegraphics[scale = 0.5]{figuras/base_datos/clasesUsuario.png}
\caption{Diagrama de clases para la funcionalidad de usuario}
\label{fig:diagramaClasesUsuario}
\end{figure}
\begin{figure}[H]
\centering
\includegraphics[scale = 0.5]{figuras/base_datos/clasesStats.png}
\caption{Diagrama de clases para la funcionalidad de las estadísticas del usuario}
\label{fig:diagramaClasesStats}
\end{figure}
\begin{figure}[H]
\centering
\includegraphics[scale = 0.5]{figuras/base_datos/clasesPartida.png}
\caption{Diagrama de clases para la funcionalidad de partida}
\label{fig:diagramaClasesPartida}
\end{figure}
\begin{figure}[H]
\centering
\includegraphics[scale = 0.5]{figuras/base_datos/clasesLiga.png}
\caption{Diagrama de clases para la funcionalidad de liga}
\label{fig:diagramaClasesLiga}
\end{figure}
\begin{figure}[H]
\centering
\includegraphics[scale = 0.5]{figuras/base_datos/clasesArticulo.png}
\caption{Diagrama de clases para la funcionalidad de artículo}
\label{fig:diagramaClasesArticulo}
\end{figure}
\begin{figure}[H]
\centering
\includegraphics[scale = 0.5]{figuras/base_datos/clasesArticuloUsuario.png}
\caption{Diagrama de clases para la funcionalidad de los artículos de usuario}
\label{fig:diagramaClasesArticuloUsuario}
\end{figure}
\subsection{Tecnologías elegidas}
\begin{itemize}
\item \textbf{Interfaz de la partida en el navegador web}. La interfaz del juego consiste en una única pantalla donde los jugadores que participan en la partida en curso se comunican. Se implementará utilizando JavaScript y Phaser. Phaser es un framework para desarrollo de juegos en HTML5 basado en la tecnología JavaScript.  \\ Se ha decidido utilizar JavaScript por la sencillez a la hora de ser visualizado en un navegador y se incorpora a la perfección con HTML5. Se ha barajado la posibilidad de utilizar Flash pero se descarta por estar obsoleta y porque algunos navegadores ya no lo soportan. También se podría haber utilizado Unity pero no se ha llevado a cabo por tener una curva de aprendizaje muy complicada.
\item \textbf{Capa de comunicación de la partida}. Es el servicio que está por debajo de la partida encargado de notificar las acciones de los jugadores al resto. Además comprueba que el transcurso de la partida es correcto, como si de un coordinador se tratara. Se utiliza lenguaje Java, ya que es un servicio Web e irá desplegado a través de un archivo .war. Además, facilita la comunicación con la tecnología WebSockets, que es la que se ha escogido para comunicar en tiempo real el navegador con el controlador. Se ha decidido WebSockets por tener una curva de aprendizaje sencilla ya que se utilizará para enviar mensajes desde el navegador al controlador. Se integra perfectamente con Java. \\
Se ha descartado utilizar Sockets.io ya que va ligado a Node.js, que es una tecnología más novedosa pero que el equipo desconoce por completo, aprenderla supone un número de horas extras y se desconoce si es una tecnologia viable para la aplicación a desarrollar.
\item \textbf{Interfaz Web}. Irá implementada en HTML 5 y utilizará JSP y Servlets para la generación de contenido dinámico y procesamiento de formularios, respectivamente. Se trata de una tecnología poco actual pero de la cuál el equipo de desarrollo tiene cierta experiencia, por lo que se asegurará la calidad del servicio.
\item \textbf{Lógica y dominio de la aplicación}. Implementado en Java para favorecer la interoperabilidad con la interfaz web.
\item \textbf{Acceso a los datos}. Se utilizarán objetos de tipo implementados en Java. De esta manera se cumple el patrón "Modelo - Vista - Controlador".
\item \textbf{Base de datos}. Se utilizará un Sistema Gestor de Bases de Datos relacional, ya que la principales consultas que se hacen son de tipo JOIN. Se ha decidido utilizar MySQL ya que es un sistema que el equipo domina. La desventaja es que es poco eficiente en comparación con otros como Oracle, pero para el número de usuarios que tendrá la aplicación es suficiente con dicho gestor.
\end{itemize}

\subsubsection{BackEnd}

Para la implementación de la lógica del juego se ha análizado el dominio del problema y se ha realizado un primer diseño de clases de análisis. Posteriormente se ha implementado objeto a objeto el diseño inicial y al mismo tiempo actualizando el diagrama de clases.

\begin{figure}[H]
\centering
\includegraphics[scale = 0.5]{figuras/logica_juego/diagramaClasesDisenyo.png}
\caption{Diagrama de clases en la fase de análisis del problema}
\label{fig:diagramaClasesLogicaJuegoInicial}
\end{figure}

\begin{figure}[H]
\centering
\includegraphics[scale = 0.5]{figuras/logica_juego/diagramaClasesImplementacion.png}
\caption{Diagrama de clases final}
\label{fig:diagramaClasesLogicaJuegoFinal}
\end{figure}

\subsection{Despliegue}
La aplicación se pondrá en marcha en un servidor Tomcat, ya que permite la instalación de aplicaciones web en formato .war. Se distinguen dos aplicaciones diferentes: la que dará soporte a la lógica de la aplicación y la encargada de coordinar una partida en curso. Ambas se comunicarán indirectamente a través de la base de datos y estarán coordinadas. \\
La base de datos será relacional ya que se necesitan muchas consultas de los datos almacenados y para cada partida hay varias inserciones o actualización de los datos almacenados. El sistema gestor de la base de datos va a ser MySQL porque se ahorran costes al ser un SGBD de código abierto y no tener que adquirir una licencia de pago. Además, el equipo cuentan con experiencia en el diseño e implementación de bases de datos utilizando MySQL. \\
La razón por la que no se ha elegido otro SGBD de código abierto son los problemas que presenta RDS Aurora. Otros SGBD como PostgreSQL son más exigentes en recursos y, por lo tanto, el coste es mayor sin repercutir un beneficio real sobre la aplicación ya que se considera que MySQL es más que suficiente para una aplicación de estas características.
El patrón de diseño utilizado para la comunicación del sistema con la base de datos es \textit{façade} ya que permite dividir en el sistema completo en dos o más subsistemas consiguiendo un alto desacoplamiento de la base de datos y el juego. Así se puede dividir mejor el trabajo de forma independiente entre los equipos para poder llevar a cabo un diseño e implementación bottom-up.

%\begin{itemize}
%\item  DIAGRAMA DE DESPLIEGUE DE LA APPLICACIÓN WEB EN 3 CAPAS
%\item \textbf{Diagramas arquitecturales (de módulos, de componentes y conectores, de distribución), patrones de diseño y estilos arquitecturales que se aplicarán. Las interfaces (de módulos y de componentes) son especialmente importantes. También lo son los protocolos de comunicación entre componentes.}
%\item \textbf{Tecnologías elegidas (lenguajes de programación, componentes que se integrarán, API web externas con las que se conectará etc.)--ya está--.}

%\item \textbf{Otros aspectos técnicos de interés (p.ej. si hay base de datos si va a ser SQL,  si algunas de las operaciones van a ser asíncronas o no, si se van a usar tecnologías web, cómo se van a considerar los requisitos de seguridad o de prestaciones, cómo y dónde se harán las instalaciones y despliegues etc.)}
%\end{itemize}
%\textbf{Hay que justificar todas las decisiones de diseño. Esto exige contestar a dos preguntas sobre cada decisión: ¿qué alternativas se barajaron? y ¿por qué se eligió una y no las otras?}

\clearpage
\section{Memoria del proyecto}
\label{memoria}
En este capítulo se describirá cómo se ha llevado a cabo el proyecto, qué cambios se han hecho respecto a la versión inicial, imprevistos surgidos, etc
\subsection{Inicio del proyecto}
\label{Inicio del proyecto}
Describir cómo transcurrió esta fase del proyecto, especialmente los resultados de llevar a cabo los procesos descritos en la sección Procesos de inicio del proyecto.
\subsection{Ejecución y control del proyecto}
\label{Ejecucion y control del proyecto}
Describir cómo transcurrió esta fase del proyecto, especialmente los resultados de llevar a cabo los procesos descritos en la sección Procesos de ejecución y control del proyecto y en la sección Procesos técnicos. No olvidar:
Cómo se ha realizado el reparto de trabajo entre miembros del equipo. Cómo ha transcurrido la comunicación interna. 
Cómo se ha medido el progreso del proyecto. Cómo se sabía el trabajo realizado, el trabajo pendiente y lo que estaba haciendo cada persona.
Los ajustes realizados cuando se detectaron divergencias frente al calendario inicial (ajustes en el trabajo y/o ajustes en el calendario). Si se han identificado las causas de estas divergencias, explicarlas.
Adecuación de las herramientas y tecnologías empleadas. Si ha habido que cambiar alguna decisión de diseño o de tecnología, y por qué.
Funcionamiento de los procesos de control de versiones del código, construcción y despliegue. ¿Ha habido problemas con las integraciones? ¿Problemas con los despliegues? ¿Se han perdido cosas por errores humanos? ¿Cómo se han abordado estas tareas?
Pruebas del software. ¿Se han podido cumplir las ideas que se tenían al respecto?

\clearpage
\section{Conclusiones}
\label{conclusiones}

Desde el inicio hasta el final del proyecto, los integrantes del equipo Margaret Hamilton hemos adquirido mucha experiencia en el proceso de gestión y control de un proyecto de tamaño medio, donde la división del trabajo, la asignación de recursos y tareas, la calidad del software y la comuncación han sido esenciales. Dada la complejidad del proyecto escogido, el desarrollo del producto ha sido una tarea muy ardua. Esto se debe a la existencia de una gran cantidad de componentes diferentes y las dependencias entre estos. Además, la integración entre estos componentes ha consumido gran cantidad de los recursos del equipo. Sin embargo, esta característica del proyecto ha servido para que el equipo adquiera experiencia en el desarrollo de aplicaciones donde la sincronización y comunicación entre de los sistemas es crítica. También se ha adquirido experiencia en el desarrollo de sistemas de inteligencia artificial donde la información del estado es incompleta y la evolución de los estados no es determinista, sistemas que están en auge y tienen un futuro prometedor.

Para la realización de futuros proyectos, gran parte de la experiencia adquirida será de gran utilidad. En el apartado de estimación de costes y planificación antes de iniciar el proyecto se ha adquirido una buena referencia que junto con otros proyectos sentarán la base de unas estimaciones cada vez más correctas. En el futuro, a la hora de escoger las tecnologías con las que desarrollar los proyectos se evitará utilizar Java debido a su bajo rendimiento y falta de seguridad. En cuanto a los procesos técnicos se ha aprendido mucho en cuanto a la definición de interfaces y la documentación. En el futuro la definición de interfaces ocupará un lugar principal en el desarrollo de paquetes. Por otro lado, el apartado de pruebas deberá tener en cuenta la integración entre los diferentes componentes y los posibles efectos colaterales en la comunicación entre estos, así como intentar realizar el despliegue en el ecosistema real lo antes posible, para evitar fallos inesperados cerca del final.
\\
Gracias a las nuevas habilidades adquiridas a lo largo de este proyecto los integrantes han logrado aumentar sus capacidades de trabajo colaborativo para mejorar el nivel de la calidad del software ofrecido.


\clearpage
\section*{Anexo I. Glosario}
\begin{description}
\item[Amazon AWS:] Amazon Web Services, plataforma cloud ofrecida por Amazon. \cite{aws}
\item[Amazon RDS Aurora:] Servicio de creación y mantenimiento de bases de datos relacionales. \cite{aurora}
\item[Amazon EC2:] Servicio de servidores privados virtuales de AWS que permite lanzar máquinas virtuales con los sistemas operativos y configuraciones deseadas. Además, permite contratar más o menos recursos bajo demanda, adaptándose así a los picos de carga. \cite{ec2}
\item[PostgreSQL:] es un sistema de gestión de bases de datos relacional orientado a objetos y libre, publicado bajo la licencia PostgreSQL.
\item[MySQL:] es un sistema de gestión de bases de datos relacional desarrollado bajo licencia dual: Licencia pública general/Licencia comercial por Oracle Corporation.
\item[Guiñote:] Juego de cartas español en el que pueden participar dos jugadores o dos parejas de jugadores. Se utiliza la baraja española de 40 cartas con cuatro palos. A continuación se explican detalladamente las reglas del juego:
\begin{itemize}
\item{El juego comienza repartiendo 6 cartas a cada jugador y colocando una carta en medio del tapete que determina el palo que es "triunfo". La manera de jugar es por rondas denominadas bazas, en las que cada jugador tira una carta. La baza se gana si se tira el triunfo más alto, o si no hay triunfo, si se tira la carta con mayor valor del palo de la carta de salida. Al ganar la baza, se suma a la puntuación de la pareja o el jugador el valor de las cartas. Además si se gana la baza se puede intercambiar el siete del palo de triunfo por la carta que se encuentra en medio del tapa si su valor es superior a la del siete, o también se puede sumar puntuación si el jugador tiene un "cante" que significa que el jugador posee la sota y el rey de un mismo palo, lo cual da 20 puntos si el palo no es triunfo y 40 puntos si el palo es triunfo.}
\item{El valor de las cartas de forma decreciente es el siguiente: As (11 pts), tres (10 pts), rey (4 pts), sota (3 pts), caballo (2 pts), siete (0 pts), seis (0 pts), cinco (0 pts), cuatro (0 pts) y dos (0 pts).}
\item{Después de cada baza, y mientras queden cartas en el mazo, cada jugador roba una carta empezando por el jugador que se haya llevado la última baza. Tras esto, será el primer jugador en jugar la primera carta de la siguiente baza.}
\item{Una vez que se acaban las cartas para robar se introducen restricciones a la hora de tirar las cartas. El primer jugador en tirar no tiene restricciones pero el resto de jugadores deberá tirar una carta del mismo palo o si no tiene, triunfo, además de que está obligado a "matar", es decir, tirar una carta para intentar ganar la baza. En caso de que el jugador no pueda cumplir ninguna de estas restricciones con las cartas que posee puede tirar la carta que desee. Finalmente, el equipo que gane la última baza ganará 10 puntos extra.}
\item{Para determinar quién ha ganado, se suma la puntuación y gana el jugador o la pareja que supere los 100 puntos (divididos habitualmente en 50 “malas”, los primeros 50 puntos, y 50 “buenas”, los restantes). En caso de que nadie supere la anterior cifra se vuelve a repartir y a cada baza que se gana se incrementa la puntuación hasta que se llega a 100, momento en el que se gana. En el caso de que ambos equipos superasen los 100 puntos en la misma jugada, ganaría aquel que hubiese ganado la última baza.}
\end{itemize}
\end{description}
\addcontentsline{toc}{section}{\protect\numberline{}Anexo I. Glosario}%
\newpage
\bibliographystyle{abbrv} % We choose the "plain" reference style
\bibliography{data}
\end{document}
