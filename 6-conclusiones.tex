\section{Conclusiones}
\label{conclusiones}

Desde el inicio hasta el final del proyecto, el equipo Margaret Hamilton hemos adquirido mucha experiencia en el proceso de gestión y control de un proyecto de tamaño medio. Dónde la división del trabajo, la asignación de recursos y tareas, la calidad del software y la comuncación son esenciales. Dada la complejidad del proyecto escogido, el desarrollo del producto ha sido una tarea muy ardua. Esto se debe a la existencia de una gran cantidad de componentes diferentes y las dependencias entre estos. Además, la integración entre estos componentes ha consumido gran cantidad de los recursos del equipo. Sin embargo, esta característica del proyecto ha servido para que el equipo adquiera experiencia en el desarrollo de aplicaciones donde la sincronización y comunicación entre de los sistemas es crítica. También se ha adquirido experiencia en el desarrollo de sistemas de inteligencia artificial donde la información del estado es incompleta y la evolución de los estados no es determinista. Dichos sistemas están en auge y tienen futuro prometedor.

Para la realización de futuros proyectos, gran parte de la experiencia adquirida será de gran utilidad. En el apartado de estimación de costes y planificación antes de iniciar el proyecto se ha adquirido una buena referencia que junto con otros proyectos sienten la base de unas estimaciones correctas. En el futuro, a la hora de escoger las tecnologías con las que desarrollar los proyectos se evitará utilizar java debido a su bajo rendimiento y falta de seguridad. En cuanto a los procesos técnicos se ha aprendido mucho en cuanto a la definición de interfaces y la documentación. En el futuro la definición de interfaces ocupará un lugar principal en el desarrollo de paquetes. Por otro lado, el apartado de pruebas deberá tener en cuenta la integración entre los diferentes componentes y los posibles efectos colaterales en la comunicación entre estos.
