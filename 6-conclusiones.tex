\section{Conclusiones}
\label{conclusiones}
A continuación se describen los objetivos principales de cada uno de los equipos de trabajo así como un pequeño resumen de cómo se ha desarrollado el trabajo de cada equipo en esta primera iteración. En el resto de la sección se describe detalladamente todo el desarrollo del proyecto.\\

\subsubsection*{Conclusiones Personales}
El principal objetivo del Frontend eran desarrollar las interfaces con las que los usuarios interaccionarían con la aplicación. Se distinguen dos interfaces, las de navegación web y la parte jugable del guiñote. La parte jugable se ha podido implementar en su mayoría, a falta del modo espectador y la parte de personalización. Sin embargo, la parte de la interfaz de navegación se completará para la segunda iteración, ya que únicamente se tienen las pantallas de login y la navegación.
\\
En cuanto al Middleware, el principal objetivo era comunicar la interfaz de juego para que los usuarios puedan jugar entre ellos, que se ha conseguido con éxito. Aun así, es necesario completar con la funcionalidad de que un usuario puede cambiar de dispositivo en una misma partida.

\subsubsection*{A mejorar}
El objetivo del equipo backend era desarrollar un paquete de java que representase la lógica del juego del guiñote, al mismo tiempo que desarrollaba gran parte de la página web dinámica del proyecto. Sin embargo, a mitad de desarrollo el trabajo, se centró en acabar la parte de la lógica del juego. Dicho objetivo se ha cumplido completamente con algunos retrasos en la entrega. Como consecuencia la web dinámica arrastra un leve retraso que deberá ser compensado en la segunda iteración. A pesar de dichas dificultades el equipo ha trabajado mucho para conseguir una primera versión funcional antes de la primera iteración.
