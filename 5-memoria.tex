\section{Memoria del proyecto}
\label{memoria}
\emph{En este capítulo se describirá cómo se ha llevado a cabo el proyecto, qué cambios se han hecho respecto a la versión inicial, imprevistos surgidos, etc}
\subsubsection{BackEnd}
El objetivo del equipo backend era desarrollar un paquete de java que representase la lógica del juego del guiñote, al mismo tiempo que desarrollaba gran parte de la página web dinámica del proyecto. Sin embargo a mitad de desarrollo el trabajo se centró en acabar la parte de la lógica del juego en 3 semanas. Dicho objetivo no se ha cumplido completamente dado que el desarrollo de este modulo se alargó media semana más. Como consecuencia la web dinámica arrastra un leve retraso que deberá ser compensado en la segunda iteración. A pesar de dichas dificultades el equipo ha trabajado mucho para conseguir una primera versión funcional antes de la primera iteración.

\subsection{Inicio del proyecto}
\label{Inicio del proyecto}
\emph{Describir cómo transcurrió esta fase del proyecto, especialmente los resultados de llevar a cabo los procesos descritos en la sección Procesos de inicio del proyecto.}
\subsubsection{BackEnd}
Durante la primera semana el equipo adquirió y configuró el software de desarrollo intellij IDEA. Tras una primera toma de contacto comenzó el trabajo. Sin embargo, a lo largo de la primera iteración el equipo se ha encontrado con diferentes problemas de integración debido a la inexperiencia del IDE. Por lo tanto en futuros proyectos sería recomendable incluir en formación la adquisición de experiencia con los IDE's.
\subsection{Ejecución y control del proyecto}
\label{Ejecucion y control del proyecto}
\emph{
Describir cómo transcurrió esta fase del proyecto, especialmente los resultados de llevar a cabo los procesos descritos en la sección Procesos de ejecución y control del proyecto y en la sección Procesos técnicos. No olvidar:
Cómo se ha realizado el reparto de trabajo entre miembros del equipo. Cómo ha transcurrido la comunicación interna. 
Cómo se ha medido el progreso del proyecto. Cómo se sabía el trabajo realizado, el trabajo pendiente y lo que estaba haciendo cada persona.
Los ajustes realizados cuando se detectaron divergencias frente al calendario inicial (ajustes en el trabajo y/o ajustes en el calendario). Si se han identificado las causas de estas divergencias, explicarlas.
Adecuación de las herramientas y tecnologías empleadas. Si ha habido que cambiar alguna decisión de diseño o de tecnología, y por qué.
Funcionamiento de los procesos de control de versiones del código, construcción y despliegue. ¿Ha habido problemas con las integraciones? ¿Problemas con los despliegues? ¿Se han perdido cosas por errores humanos? ¿Cómo se han abordado estas tareas?
Pruebas del software. ¿Se han podido cumplir las ideas que se tenían al respecto?}
\subsubsection{BackEnd}

